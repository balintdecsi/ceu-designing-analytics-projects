\documentclass[11pt,a4paper]{article}
\usepackage[utf8]{inputenc}
\usepackage[T1]{fontenc}
\usepackage{geometry}
\usepackage{hyperref}
\usepackage{parskip}

\geometry{margin=0.75in}

\title{\textbf{Week 1: Capstone Search Kickoff}}
\author{Bálint Décsi}
\date{January 23, 2026}

\begin{document}

\maketitle

\section*{Where I'm Looking \& Conversations So Far}

My search focuses on two directions: the public sector and the telecommunications industry. The most advanced lead is with a Budapest district government involved in an EU-funded climate resilience project. They have an existing codebase (\texttt{https://github.com/CLIMAAX}) and a v1.0 prediction model. I've discussed developing a "v2.0" iteration with their project leads, where better predictions could directly influence population well-being.

Parallel to this, I am leveraging my role at Deutsche Telekom to explore opportunities within DT and Magenta Telekom. This path utilizes my field expertise, though conversations are less mature. I am currently identifying internal projects that fit the academic scope.

\section*{What Makes a Good Capstone \& My Biggest Challenge}

My primary challenge is distinguishing between a "data project" and a "decision project." For the government proposal, data availability is excellent, but I must ensure the goal isn't just "improve accuracy." I need to define the specific \textit{decision} (e.g., budget allocation, alert thresholds) that changes if model performance improves.

For the Telco path, the challenge is finding a problem requiring predictive modeling rather than just rules or dashboards. My fear is spending months on data cleaning only to find a simple heuristic suffices. Securing a clear target variable and a stakeholder who distinguishes "analytics" from "reporting" is the main hurdle.

\section*{Messy Situation Practice}

\textit{Scenario: Sponsor says, "We have tons of data. Just explore it and find interesting things."}

\textbf{Response:}

"Thank you for the access. To ensure I deliver actionable insights rather than just trivia, could we identify a specific business decision you are facing or a metric that is underperforming? Anchoring the analysis to a concrete problem will help me cut through the noise and focus on finding value you can actually use."

\section*{Reflection}

This week highlighted my tendency to prioritize technical feasibility (e.g., "is the data there?") over business impact. I realized I am more comfortable discussing architectures than digging for the "why." Moving forward, I will consciously start conversations with the business problem to avoid technically perfect but useless solutions.

\end{document}
